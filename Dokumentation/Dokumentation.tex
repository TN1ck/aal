
\documentclass[10pt,a4paper]{report}
\usepackage[utf8]{inputenc}
\usepackage[german]{babel}
\usepackage{amsmath}
\usepackage{amsfonts}
\usepackage{amssymb}

\usepackage{listings}
\usepackage{hyperref}

\newcommand{\code}[1]{%
{\fontfamily{cmvtt}\selectfont #1}}

\usepackage[left=2cm,right=2cm,top=2cm,bottom=2cm]{geometry}
\begin{document}
% \title{Dokumentation Projekt AAL}
% \author{Tom Nick\\Jonathan Seilkopf\\Niklas Gebauer\\Maximilian Bachl\\Tom Lehmann}
% %\subtitle{Applikationsgruppe}
% \maketitle
% \newpage


\begin{titlepage}

\newcommand{\HRule}{\rule{\linewidth}{0.5mm}} % Defines a new command for the horizontal lines, change thickness here

\center % Center everything on the page
 
%	HEADING SECTIONS

\textsc{\LARGE Technische Universität Berlin}\\[2.5cm] % Name of your university/college
\textsc{\Large Projekt AAL}\\[0.5cm] % Major heading such as course name
\textsc{\large Applikationsgruppe}\\[0.5cm] % Minor heading such as course title


%	TITLE SECTION

\HRule \\[0.4cm]
{ \textsc{\Huge Dokumentation}}\\[0.4cm] % Title of your document
\HRule \\[1.5cm]
 
%	AUTHOR SECTION

% \begin{minipage}{0.4\textwidth}
% \begin{flushleft} \large
% \emph{Author:}\\
% John \textsc{Smith} % Your name
% \end{flushleft}
% \end{minipage}
% ~
% \begin{minipage}{0.4\textwidth}
% \begin{flushright} 
% \large
% \emph{Supervisor:} \\
% Dr. James \textsc{Smith} % Supervisor's Name
% \end{flushright}
% \end{minipage}\\[4cm]

% If you don't want a supervisor, uncomment the two lines below and remove the section above
%\Large \emph{Author:}\\
Tom Nick\\
Jonathan Seilkopf\\
Niklas Gebauer\\
Maximilian Bachl\\
Tom Lehmann\\
[3cm] % Your name

%	DATE SECTION

{\large \today}\\[3cm] % Date, change the \today to a set date if you want to be precise

\vfill % Fill the rest of the page with whitespace
\end{titlepage}



\tableofcontents


\chapter{Entwicklerhandbuch}
	\section{Installation}
	Für die vollständige Lauffähigkeit unserer finalen Abgabe, müssen folgende Programme installiert sein:
	\begin{itemize}
		\item \href{http://www.playframework.com/}{Play}
		\item \href{http://nodejs.org/}{node.js} 
		\item \href{http://www.oracle.com/technetwork/java/javase/downloads/java-se-jre-7-download-432155.html}{Java 1.7} 
		\item \href{http://www.google.de/intl/de/chrome/browser/}{Google Chrome}
	\end{itemize}
	Zum Starten des Projekts muss zunächst das Play-Backend gestartet werden. Das kann getan werden, indem aus dem Projektverzeichnis heraus Play mit
	\code{play run} via Konsole gestartet wird. Nachdem der Server fertig geladen hat, muss die Seite einmalig über die Adresse \code{http://localhost:9000} gestartet werden.
	Bei diesem Aufruf werden die Jiac-Agenten initialisiert und gestartet. Wird die Seite mehrmals über diese URL geladen, werden die Jiac-Agenten mehrfach gestartet. Das kann zu undefiniertem Verhalten führen und sollte deshalb vermieden werden. Nach dem ersten Aufruf, wechselt der Status zu \code{http://localhost:9000/index.html\#/nouser}. Ab sofort reagiert die Wall auf einkommende Nachrichten und ändert ihren Status selbstständig.

	\section{Projektstruktur}
		\subsection{Allgemeiner Aufbau}
			Aus diversen Gründen haben wir uns dazu entschieden das Frontend mit dem, von Google entwickelten, Javascript-Framework AngularJS\footnote{\href{http://angularjs.org/}{http://angularjs.org/}} zu entwickeln. die komplette Frontendimplementierung befindet sich im Unterordner \code{public/angular/app}. Die Widgets haben wir als Angular-Directives implementiert und diese befinden sich im \code{scripts/directives}-Ordner. Allgemeine Funktionen, welche die gesamte Applikation beziehungsweise den gerade relevanten Teil der Applikation betreffen, werden in den Controllern realisiert. Für häufig genutzte und ausgliederbare Funktionalität, benutzen wir die Services. Ein weiterer zentraler Bestandteil unserer Applikation ist der AngularUI Router\footnote{\href{http://github.com/angular-ui/ui-router}{http://github.com/angular-ui/ui-router}} welcher für die Anzeige und den Wechsel der einzelnen Zustände zuständig ist. Sämtliche visuell relevanten Codeteile befinden sich in dem Unterordner \code{views}.\\\\
			Unser Backend-Code ist im \code{app}-Verzeichnis abgelegt. Die Aufgabe des Backends  besteht im wesentlichen darin, sich um die Kommunikation mit anderen Gruppen des Projekts via Jiac zu kümmern und das Frontend mit Daten zu versorgen. Weiterhin stellt es der Wallapplikation sowie den Mobilgeräten, welche zur Bedienung ebenjener verwendet werden die Websockets als Kommunikationskanal zur Verfügung.

		\subsection{Frontend} 
	
	\section{Erweiterung}

\chapter{Nutzeranleitung}

\chapter{Projektbericht}

\end{document}